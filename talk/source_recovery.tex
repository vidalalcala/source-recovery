\documentclass{beamer}
\mode<presentation>
\usetheme{Madrid}

\usepackage{hyperref}
\usepackage{fancyvrb}
\usepackage{verbatim}



\usepackage[utf8]{inputenc}
%\usepackage[spanish]{babel}

% set fonts for nicer pdf view
\IfFileExists{lmodern.sty}{\usepackage{lmodern}}{}

\usepackage{graphicx}
\setbeamertemplate{navigation symbols}{}

\newcommand {\framedgraphic}[1] {
        \begin{center}
            \includegraphics[width=\textwidth,height=0.8\textheight , 
            keepaspectratio]{#1}
        \end{center}
}

%% Place here your \newcommand's and \renewcommand's. Some examples already included.
%%
\renewcommand{\le}{\leqslant}
\renewcommand{\ge}{\geqslant}
\renewcommand{\emptyset}{\ensuremath{\varnothing}}
\newcommand{\ds}{\displaystyle}
\newcommand{\R}{\ensuremath{\mathbb{R}}}
\newcommand{\Q}{\ensuremath{\mathbb{Q}}}
\newcommand{\Z}{\ensuremath{\mathbb{Z}}}
\newcommand{\N}{\ensuremath{\mathbb{N}}}
\newcommand{\T}{\ensuremath{\mathbb{T}}}
\newcommand{\eps}{\varepsilon}
\newcommand{\closure}[1]{\ensuremath{\overline{#1}}}

%MINE
\newcommand{\norm}[1]{\left\Vert#1\right\Vert}
\newcommand{\abs}[1]{\left\vert#1\right\vert}
\newcommand{\set}[1]{\left\{#1\right\}}
\newcommand{\epsi}{\varepsilon}
\newcommand{\To}{\longrightarrow}
\newcommand{\BX}{\mathbf{B}(X)}
\newcommand{\A}{\mathcal{A}}
\newcommand{\St}{\mathcal{S}}
\newcommand{\Par}{\mathcal{P}}
\newcommand{\La}{\mathcal{L}}
\newcommand{\F}{\mathcal{F}}

%LINEAR ALGEBRA
\newcommand{\tr}{\mathrm{tr}}

%PROBABILITY
\newcommand{\Var}{\mathrm{Var}\:}
\newcommand{\dis}{\mathrm{d}}


%CALCULUS OF VARIATIONS
\usepackage{accents}
\newcommand*{\dt}[1]{%
  \accentset{\mbox{\large\bfseries .}}{#1}}
\newcommand*{\ddt}[1]{%
  \accentset{\mbox{\large\bfseries .\hspace{-0.25ex}.}}{#1}}

%\newcommand{\acim}{\textsc{acim}\xspace}
%\newcommand{\acims}{\textsc{acim}s\xspace}

%%
%% Place here your \newtheorem's:
%%

%% Some examples commented out below. Create your own or use these...
%%%%%%%%%\swapnumbers % this makes the numbers appear before the statement name.
%\theoremstyle{plain}
%\newtheorem{thm}{Theorem}[chapter]
%\newtheorem{prop}[thm]{Proposition}
%\newtheorem{lemma}[thm]{Lemma}
%\newtheorem{cor}[thm]{Corollary}

%\theoremstyle{definition}
%\newtheorem{define}{Definition}[chapter]

%\theoremstyle{remark}
%\newtheorem*{rmk*}{Remark}
%\newtheorem*{rmks*}{Remarks}

%% This defines the "proo" environment, which is the same as proof, but
%% with "Proof:" instead of "Proof.". I prefer the former.
%\newenvironment{proo}{\begin{proof}[Proof:]}{\end{proof}}


\title[Source and velocity recovery from one time measurements in the advection diffusion equation]{Source and velocity recovery from one time measurements in the advection diffusion equation} 
\subtitle{Inferencia de parámetros en PDE's}
\author{José Vidal Alcalá Burgos}
\institute{Catedrático CONACYT - CIMAT}
\date{November 5 2015} 

\begin{document}
% this prints title, author etc. info from above
\begin{frame}
\titlepage
%\centerline{\includegraphics[height=1.1in,width=4in]{Preview-TempPDF.pdf}}
\end{frame}


\begin{frame} 
\frametitle{Advection diffusion equation}
We can model the propagation of contaminants in a medium with the advection diffusion equation.
\begin{equation}\label{forward}
  \begin{split}
    u_t(x, t) &= \Delta u(x, t) + \psi \cdot \nabla u(x,t) + f(x)\\
    u(x, 0) &= 0\:,
  \end{split}
\end{equation}
where
\begin{itemize}
  \item $u(x,t)$: contaminant density
  \item $\psi$: \alert{constant} velocity of the medium
  \item $f(x)$: contaminant \alert{source} intensity in the position $x$
\end{itemize}

\alert{QUESTION:} How many time measurements $u(\cdot, T_i)$ do we need to recover $f$ and $\psi$ uniquely?
\end{frame}

\begin{frame} 
\frametitle{Uniqueness result}

If we only allow sources $f$ with \alert{compact support}, then \alert{one time} measurement $u(x,T)$ is enough. The main argument is the following:
\begin{enumerate}
  \item If we have solutions $u_k$ of the equation \eqref{forward} with parameters $\psi_k$, $f_k$ such that $\psi_1 \neq \psi_2$ and $u_1(\cdot, T)=u_2(\cdot, T)$, then
    \begin{equation*}
      \begin{split}
        \widehat{f}_{1}(\xi) &= \widehat{f}_2(\xi) \frac{a_1}{a_2}\frac{(e^{a_2 T} - 1)}{(e^{a_1 T} - 1)},\quad
        a_{k} = (i \psi_k - 2 \pi \xi)(2 \pi \xi), \quad \xi\neq 0 \:.
      \end{split}
    \end{equation*}
  \item $\widehat{f}_{k}$ is an \alert{entire} function.
  \item $\widehat{f}_{k}$ has at most \alert{$O(r)$ zeros} inside a ball of radius $r$.
  \item $(e^{a_1 T} - 1)$ has \alert{$O(r^2)$} zeros inside a ball of radius $r$, all of them different from the zeros of $(e^{a_2 T} - 1)$.
\end{enumerate}

\end{frame}

\begin{frame}
\frametitle{Adjoint equation}
Given the \alert{data} $u_d(\cdot) = u(\cdot, T)$, we want minimize (over $\psi$ and $f$)
\begin{equation*}
  \begin{split}
    F(u) = \frac{1}{2} \int_{\R} (u(x,T) - u_d(x))^2 dx\:,
  \end{split}
\end{equation*}
where $u$ satisfies the restriction
\begin{equation}\label{restriction}
	\begin{split}
	  0 &= g(u, \psi, f) = 
	  \left( \begin{array}{c}
          u_t - u_{xx} - \psi u_x - f(x) \\
          u(x, 0) \end{array} \right)\:.
	\end{split}
\end{equation}
We calculate
\begin{equation*}
  \begin{split}
    F_{u} \dot{u} &= \int_{\R} (u(x, T) - u_{d}(x)) \dot{u}(x, T) dx, \quad
    g_{u} \dot{u} = \left( \begin{array}{c}
                                    \dot{u}_t - \dot{u}_{xx} - \psi \dot{u}_x\\
                                    \dot{u}(x, 0) \end{array} \right)\\
    g_{\psi} \dot{\psi} &= \left( \begin{array}{c}
                                    - u_x \dot{\psi}\\
                                    0 \end{array} \right), \quad
    g_{f} \dot{f} = \left( \begin{array}{c}
                                    - \dot{f}\\
                                    0 \end{array} \right)
  \end{split}
\end{equation*}
\end{frame}

\begin{frame}
\frametitle{Adjoint equation}
The \alert{adjoint} vector $\lambda$ is the solution of the \alert{adjoint equation}
\begin{equation}\label{adjoint}
  \begin{split}
    F_{u} \dot{u} &= - \biggl< \left( \begin{array}{c}
          \lambda(x, t) \\
          \lambda_{0}(x) \end{array} \right), g_{u} \dot{u}\biggr>_{L^2}	
  \end{split}
\end{equation}
The chain rule and the adjoint equation give
\begin{equation*}
  \begin{split}
   d_{f} F \dot{f} &= F_{u} d_{f} u \dot{f} = - \biggl< \left( \begin{array}{c}
          \lambda(x, t) \\
          \lambda_{0}(x) \end{array} \right), g_{u} d_{f} u \dot{f}\biggr>_{L^2}\:,	 
  \end{split}
\end{equation*}
and taking the derivative with respect to $f$ of the restriction we obtain
\begin{equation*}
  \begin{split}
    g_{u} d_{f} u \dot{f} + g_f \dot{f} = 0\:.
  \end{split}
\end{equation*}
Therefore
\begin{equation*}
  \begin{split}
    d_f F \dot{f} & = \biggl< \left( \begin{array}{c}
          \lambda(x, t) \\
          \lambda_{0}(x) \end{array} \right), g_{f} \dot{f}\biggr>_{L^2}
    = \biggl< \left( \begin{array}{c}
          \lambda(x, t) \\
          \lambda_{0}(x) \end{array} \right),\left( \begin{array}{c}
                                    - \dot{f}\\
                                    0 \end{array} \right)\biggr>_{L^2}\\
    d_{\psi} F \dot{\psi} &= \biggl< \left( \begin{array}{c}
          \lambda(x, t) \\
          \lambda_{0}(x) \end{array} \right),\left( \begin{array}{c}
                                    - u_{x} \dot{\psi}\\
                                    0 \end{array} \right)\biggr>_{L^2}
  \end{split}
\end{equation*}

\end{frame}

\begin{frame}
\frametitle{Adjoint equation}
Integration by parts shows that the adjoint equation can be written as
\begin{equation}\label{adjoint_pde}
	\begin{split}
	\lambda_t(x,t) &= - \lambda_{xx}(x,t) + \psi \lambda_x(x,t) \\
	\lambda(x, T) &= u_{d}(x) - u(x, T) \\
	\lambda(x,0) &= \lambda_0(x)
	\end{split}
\end{equation}
\alert{Summary:} Given any $\psi$ and $f$
\begin{enumerate}
  \item Solve for $u(x, t)$ in the advection diffusion equation.
  \item Use $u(x, T)$ and the data $u_{d}(x)$ to find the solution $\lambda$ of the adjoint equation.
  \item Use $u$ and $\lambda$ to evaluate
  \begin{equation*}
	\begin{split}
	  d_f F & = -\int_{0}^{T} \lambda(x, t) dt, \quad d_\psi F = -\int_{0}^{T} \int_{\R} \lambda(x,t) u_{x}(x, t)dxdt\\
	\end{split}
\end{equation*}
\end{enumerate}

\end{frame}

\begin{frame}
\frametitle{Numerical implementation}

Now we can minimize $F$ over $\psi$ and $f$ with a simple gradient descent:
\begin{equation*}
	\begin{split}
	f_{n+1} &= f_n - \alpha d_{f} F, \quad \psi_{n+1}= \psi_{n} - \beta d_{\psi} F
	\end{split}
\end{equation*}
\end{frame}






























\end{document}
